\documentclass{scrartcl}
\usepackage[ngerman]{babel}
\usepackage[utf8]{inputenc}
\usepackage{cmbright}
\usepackage{parskip}
\usepackage{csquotes}

\begin{document}
\thispagestyle{empty}
\section*{Professionelles Testen für Python mit pytest}

Folgende Programmpunkte sind für den Workshop geplant:

\begin{itemize}
\item Einleitung und Terminologie: Warum automatisierte Tests, Arten von Tests, Aufgaben eines Test-Frameworks.
\item Tests schreiben mit pytest: Installation, wichtige Features, Konfiguration, Assertions.
\item Tests organisieren: \enquote{markers}, Tests parametrisieren mit Daten, Tests überspringen.
\item Abhängigkeiten modularisieren mit \enquote{fixtures}: Setup/Teardown, Dependency Injection.
\item Abhängigkeiten vermeiden mit Patching/Mocking: Warum, wann, wann lieber nicht, und wie.
\item Mit bestehenden Testsuites und Frameworks umgehen: Existierende unittest/nose/Django-Testsuites ausführen, Strategien bei der Migration zu pytest.
\item Plugins: Automatisch Testdaten generieren mit \enquote{hypothesis} (property-based testing), Testabdeckung (coverage), Nutzung mehrerer CPU-Cores/Maschinen und vieles mehr.
\item Eigene Plugins verfassen: Verfügbare Hooks, Plugins als Teil einer Testsuite, Paketierung und Verteilung.
\item Offene Fragerunde: Platz für Fragen und Probleme bei der Integration von pytest in bestehende Projekte der Teilnehmenden.
\item Je nach Zeit und Interessen des Publikums kann auch auf weitere Themen zu pytest oder zu verwandten Projekten (tox/devpi) eingegangen werden.
\end{itemize}

Der Workshop ist als \enquote{Hands-on}-Kurs konzipiert und enthält viele Übungen, womit die Teilnehmenden von Anfang an ihre eigenen Tests mit pytest schreiben können.
\end{document}
